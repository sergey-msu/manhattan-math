\chapter{Вглубь атома}

\epigraph{\emph{TODO}}{TODO}

В книгах по математическим вопросам ядерной физики обычно принято без долгих предисловий переходить к моделям и методам их исследования.
При этом изложение ведется на принятом в математике высоком уровне строгости.
Это имеет свой смысл.
Дело в том, что оформившаяся математическая теория того или иного физического процесса вполне может быть изложена отдельно от своего физического контекста.
К сожалению, такого рода книги обычно носят довольно академический характер и зачастую понятны только узкому кругу специалистов, имеющих должный уровень математической подготовки. Для первого погружения в задачу, такой подход подходит плохо.

Математику, стоящую за современной ядерной физикой, не стоит на начальном этапе пытаться рассматривать в отрыве от самого ее предмета - атома.
Для полного понимания используемых математических моделей крайне полезно хотя бы в общих чертах понимать как физику атома, так и полные проб и ошибок пути, которые привели ученых к их ключевым идеям.
  
На протяжении всей книги нам не потребуются ни глубокие факты современной ядерной физики, ни продвинутый аппарат квантовой механики.
Вполне достаточно будет общих фактов о строении атома и ядерных реакциях.
Без этого было бы крайне тяжело понять, почему те или иные задачи ставились перед математиками Манхэттенского проекта.

В данной главе мы вкратце проследим историю самой идеи об атоме начиная с древних времен вплоть до принятой сегодня точки зрения. 
Читатель, знакомый с основами физики атома, может пропустить эту главу и сразу перейти к математической сути вопроса в гл. ??? 

\section*{Первые идеи}

С древнейших времен не прекращались попытки человека проникнуть в суть материальных объектов, мысленно разбив их на минимально возможные части.
Вообще говоря, это универсальный метод исследования любых сложных объектов - пытаться разложить их на более простые, желательно элементарные части, понять что каждая из себя представляет и как они взаимодействуют друг с другом.
Поэтому не столь удивительно, что ученые древности пытались найти и описать первоэлементы, из которых могут состоять все объекты окружающего мира.
История этих попыток насчитывает более двух с половиной тысяч лет.

Идея существования мельчайших, далее неделимых частичек мироздания появлялась в разных религиозно-филосовских школах древней Греции, Индии и стран Востока, играя в них основополагающую роль. Атомизм в древнеиндийской философии и буддизме, по некоторым данным, возник раньше древнегреческого. Однако именно в древней Греции идеи атомизма заиграли всеми красками, глубоко проникая и давая плоды в философии, физике и особенно математике. 

Истоки древнегреческого атомизма стоит искать у Пифагора и его учеников в VI веке до н. э. 
О тех временах осталось не так много достоверных свидетельств.
Вымыслы и легенды практически не отделимы от реальных событий.
Более-менее достоверно известно, что в центр своего учения пифагорейцы ставили целые числа.
Они считали, что все объекты Вселенной должны описываться ими или их отношениями - дробями.
По легенде эта атомистическая числовая концепция была опровергнута одним из пифагорейцев - Гиппасом, впервые нашедшим несоизмеримые отрезки, то есть отрезки, отношение длин которых не равно никакому отношению двух целых чисел.
Неизвестно, какие именно это были отрезки - диагональ и сторона квадрата, правильного пятиугольника или что-то еще.
Также достоверно неизвестно наказание, которое Гиппас понес за свое открытие - смерть или простое изгнание.   

Так или иначе, атомистическая концепция в числах продержалась недолго, но натолкнула на правильные мысли других мыслителей античности - Левкиппа и Демокрита.
Они развили идею атомизма по отношению уже не к числам, а к объектам реального мира.
Про Левкиппа известно немного помимо того, что он был учителем Демокрита - основного создателя древнегреческого атомизма.
Некоторые историки вообще ставят под сомнение его существование, настолько мало информации о нем сохранилось.

Главные идеи атомизма Демокрита можно выразить в следующих положениях:
\begin{enumerate}
    \item Вселенная состоит из \textit{пустоты} и \textit{атомов}.
    \item Есть бесконечно много различных, но однородных и непроницаемых \textit{атомов}.
    \item \textit{Атомы} взаимодействуют друг с другом путем случайных столкновений, являющихся причиной всех остальных движений. 
    \item Все объекты мира стостоят из конечного числа \textit{атомов}.
    \item Никакие объекты не возникают из ничего, только путем комбинации других объектов.
    \item Никакие объекты не исчезают бесследно, только распадаются на другие объекты.
\end{enumerate} 

Таким образом, идея атомизма, рассорившая адептов крупнейшей античной математической школы, послужила хорошей почвой для идей в естествознании. 
Примечательно, что через некоторое время та же идея опять с успехом использовалась, причем никем иным, как самим Архимедом.

Здесь стоит различать атомизм математический и философский.
Исторически идеи атомизма впервые появились и применялись к объектам реального мира и носили именно философский характер.
В абстрактной науке математике и ``атомы'' представлялись как мельчайшие абстрактные величины. Они служили вполне конкретной цели - с их помощью вычислялись площади и объемы тел.
Позже мы увидим (см. главу ?????), что это вообще по сути единственный возможный способ вычисления \textit{меры} (площади или объема) сложного объекта - сколь угодно точно приблизить его более простыми объектами, мера которых известна.
Именно так определяют меру объектов на плоскости и в пространстве в современной математике.

--------------- Великий ученый античности получил прекрасное образование в Александрии - одном из крупнейших научных центров того времени. 
Там он среди прочего изучал труды Демокрита и Евдокса, откуда и узнал об идее атомизма в ее математической и физической интерпретациях.
Фактически только из трудов Архимеда и можно узнать об античном атомизме, так как все, что было до него, по большей части утеряно.
Сам Архимед использовал концепцию математического атомизма в виде идеи равносоставленности, предложенной ранее Евдоксом и значительно усовершенствованной им самим.

Идею можно проиллюстрировать на следующем примере (см. рис. ).
--------------------

К сожалению, идея равносоставленности в приведенной выше формулировке годится только отдельных случаев для и в целом неверна. 
Несложно придумать, как говорят в математике, \textit{контрпример} рассуждений подобного рода, но приводящим к совершенно неверным результатам.
Интересно, что и сам Архимед не относился к этому методу как к строгому, вероятно зная, что он может привести к ошибкам.
Архимед использовал его скорее как способ угадать правильный ответ, который затем тщательно проверял. 

-------------- Отношение к атомизму как конкретному математическому инструменту (пусть и не идеальному) позволило Евдоксу и Архимеду заложить основы современного интегрального исчисления, а вместе с ним и значительной современной математики.

Наконец, отдельного упоминания заслуживают атомистические идеи в восточной философии, родившиеся, по некоторым источникам, даже раньше античных и повлиявшие на них.
Впрочем, основные концепции там еще более туманны и представляют скорее историко-философский интерес.

Идеи раннего атомизма, конечно же, довольно далеки от современного взгляда на вещи.
И тем не менее кажется удивительным то, насколько удалось продвинуться в своих размышлениях философам древности, не обладавшим серьезными измерительными приборами.
При должной фантазии в филосовских учениях восточной и древнегреческой философии вполне можно увидеть намеки на современное представление о микромире.
В буддизме, например, с древних времен фактически фигурировал вакуум как материальная среда, в которой мельчайшие частицы непрерывно возникают и исчезают. 
Сегодня подобный взгляд на вещи составляет основу квантовой теории поля, согласно которой так называемые \textit{виртуальные частицы} ведут себя схожим образом. 
Атомы по Демокриту участвовали в непрерывных случайных столкновениях друг с другом, порождая всевозможные движения объектов. 
Последнее весьма похоже на описание броуновского движения, если считать атомы Демокрита современными молекулами.

Идеи философского атомизма были довольно популярны в древнем мире, но в итоге исчерпали себя ввиду невозможности пойти дальше умозрительной философии.
В средневековье атомизм как учение находился под запретом и чуть не был утерян вовсе.
Лишь в XVII веке к этим идеям вернулись вновь.


\section*{Время новой физики}

Серьезный прорыв был сделан в начале XIX века английским химиком Джоном Дальтоном, фактически возродившим атомизм. 
Дальтон сформулировал понятие атома, довольно близкое к современной точке зрения. 
Дальтон считал (и это верно), что атомы участвующих в химических превращениях веществ лишь перераспределяются, но не распадаются на части и не создаются. 
Таким образом, впервые возникла идея об атоме как \textit{химически неделимом} элементе мироздания, участвующем в любом превращении веществ в природе.

\section*{Строение атома}


\section*{Цепные реакции}


------------------------ IDEAS ------------------------ 


Тысячелетнее стремление человека проникнуть внутрь атома
Стремление человека познать микро- и макромир. Атомизм - Греция, Индия
Стало понятно, что пришло время совершенно новой науки.
Основные достижение физики начала XX века: история
Ущемленная роль математики, причины 
Основные достижение математики начала XX века: история. Мало кому что понятно :)
Война как катализатор науки.
Второе великое объединение: физика - использует математику
В некотором роде венец и черта - ПМ
Проект "Манхэттен": [кратко]
    история
    амбициозность
    результаты
    использование лучших мировых ученых
    результаты
    влияние на науку и технику в будущем
Советский атомный проект - тоже сила мысли и пр.
Описание источников - другие книги и пр.
Описание глав
Пара слов о задачах и ответах



Атомизм как философское учение занимает особое место в философии древней Греции.
Воображение античных философов и их умение делать обобщения на основе наблюдаемых явлений природы позволило предвосхитить даже некоторые открытия недавнего прошлого.
Так философ Демокрит, один из основоположников древнегреческого атомизма, полагал, что вселенная состоит из \textit{пустоты} и \textit{атомов}. Атомы по Демокриту могут иметь разнообразную форму, в совокупности составляют отдельные объекты мира и участвуют в непрерывных столкновениях друг с другом, порождая всевозможные движения объектов.
Таким образом, Демокритом было фактически описано броуновское движение, если считать его атомы современными молекулами.

В древней индийской философии также в свое время родились и развивались концепции атомизма, причем, вероятно, даже раньше древнегреческих.
При всем многообразии конкурирующих религиозных школ и различии в своих учениях практически все они единогласно принимали концепцию атомизма в том или ином ее виде.
В буддизме идеи атомизма традиционно понимались гораздо шире, фактически основывая свое учение на понятии \textit{дхармы} как элементарной и неделимой сути объектов - центральном для всей буддийской философии.




XX век был поистине богатым научными открытиями в самых разных областях науки. Ученые как никогда приблизились к пониманию механики как микро, так и макро процессов окружающего мира. В биологии был обнаружен и описан основной строительный блок всего живого - молекула ДНК. Стремительно начала развиваться генная инженерия, находя приложения в самых разных отраслях человеческой деятельности. В физике были открыты общая и специальная теории относительности, квантовая механика. Выдающиеся достижения физиков и биологов активно освещались в прессе и практически сразу становились предметом жаркого обсуждения даже людьми, далекими от мира науки и в лучшем случае довольно приблизительно понимающими, о чем идет речь. 
Подобного, к сожалению, нельзя сказать об отношении к достижениям математики XX века - кроме самих математиков и, пожалуй, некоторых физиков, о них не знал практически никто. А они были поистине впечатляющими, вполне сравнимыми по потенциальной мощи с квантовой механикой или открытием ДНК. Стоит упомянуть хотя бы появление и активное использование компьютеров, необходимых для сложных расчетов тогда и распространенных повсеместно сейчас.

Отсутствие должного освещения открытий математики отчасти связано с самой спецификой данной науки. Лишь в редких случаях по-настоящему сложную математическую теорию можно объяснить широкому кругу людей-непрофессионалов. Чувство красоты математических рассуждений, доказательств и окончательных выводов необходимо упорно воспитывать в себе некоторое время, прежде чем появится понимание того, что стоит за длинными формулами и придет осознание того, как полученные выводы можно применить на практике.
Данная книга призвана восполнить этот пробел и рассказать, какую роль сыграли математики в знаменитом манхэттенском проекте, явившим миру всю мощь ядерной энергии. Я попытаюсь осветить мат. аппарат, который использовался при расчетах, связанных с конструированием атомной и водородных бомб, уделяя особое внимание методам, созданным именно в процессе работы над проектом “Манхэттен”.

Книга рассчитана на широкий круг читателей, интересующихся математикой и ее приложениями в ядерной физике. Книга будет интересна студентам и аспирантам физико-математических специальностей, а также просто интересующиеся тематикой атомной физики начала-середины XX века и применяемого там мат. аппарата. От читателей в большинстве случаем требуются лишь общие знания об основных понятиях математики - множествах и отображениях. Для понимания наиболее сложных моментов книги будет полезна специальная подготовка в рамках не ниже 2 курса физико-математических специальностей, общие знания по математическому анализу, теории вероятностей, дифференциальным уравнениям и функциональному анализу.






Формулу $E = mc^2$ как мантру может повторить практически любой современный человек.
Многие из нас так или иначе слышали о ней еще в детстве, не подозревая, что же она в действительности означает.
....


Попытки разобраться в сути какого-либо уже исследованном кем-то ранее явлении реального мира чем-то напоминают процесс очистки гипотетического фрукта с многослойной кожурой.
Первым и самым простым слоем являются личный опыт, мнения других людей и ``авторитетных'' источников о данном вопросе. 
На этом, собственно, можно и остановиться, сказав, что достаточно разобрались в вопросе.

Если полученные ответы нас не устраивают, не понятны, либо не полны и желание разобраться в сути явления не угасло, то придется перейти к следующему слою - предметной области явления, например, физике.
Необходимо хотя бы в общих чертах понять, что же именно происходит в интересующем нас явлении природы. 
Какие объекты в нем участвуют и по каким правилам взаимодействуют друг с другом. Какие моменты существенны, а какими можно пренебречь.
Продвинувшись в понимании физической сути процесса, мы 

Наконец, последний и традиционно самый трудный слой - математика явления.
Каждое явление природы имеет свой язык описания  ...  сложно .. вместо объектов - абстракции, вместо простых правил взаимодействия - сложные уравнения.



атомы - интуиция еще со времен древних греков, но дальше - перерыв почти на x000 лет связанный с тем, что увидеть объект своих измышлений уже не возможно.
прорыв - с появлением соответсвующих средств измерений, но тут ученых ждал очень большой сбрприз
до этого схема научных открытий в большинстве своем состояла в следующем - смотрели, измеряли, придумывали теорию основнную на уже известных аналогиях, потом совершенствовали приборы, и снова смотрели и придумывали анлогию и т.п.
в атомной физике известных аналогий не нашлось. Наблюдения зачастую в корне противоречили известным фактам о макромире. Любая попытка смотреть на макро-аналогии заканчивалась появлением множества противоречий теории с кспериментом и в конце концов полным провалом 


физика - ранее умение делать открытия зависело от того, насколько наблюдателен был ученый, насколько хорошо он умел проводить параллели между уже известными ялениями и только изучаемыми.
Движения огромных небесных тел описывалось исходя из аналогичных движений, которые можно было повторять в удобном масштабе в своей алборатории и т.п. [еще примеры]
Новая физика потребовала от ученых вообразить нечто не имевшее аналогов с ранее изученным в принципе. 
Это восхищало даже далеких от физики современников.
В математике такие штуки привыкли проворачивать довольно давно. 
Стефан Банах, один из создателей современной математики в ее [современном] виде, говорил "Хорошие математики видят аналогии, лучшие могут видеть аналигии между аналогиями". Сам он, безусловно, был одним из лучших.

 
слова "теория относительноси", "квантовая механик" носились в воздухе. Их можно было слышать  понимающих и истолковыва

----------------------------



https://ru.wikipedia.org/wiki/%D0%AD%D0%BB%D0%B5%D0%BA%D1%82%D1%80%D0%BE%D0%BD%D0%BD%D0%B0%D1%8F_%D0%BF%D0%BB%D0%BE%D1%82%D0%BD%D0%BE%D1%81%D1%82%D1%8C
{
В качестве модели состояния электрона в атоме, в квантовой механике принято представление об электронном облаке, плотность соответствующих участков которого пропорциональна вероятности нахождения там электрона.

Электронное облако часто изображают в виде граничной поверхности. При этом обозначение электронной области при помощи точек опускают. Пространство вокруг ядра, в котором наиболее вероятно пребывание электрона, называют атомной орбиталью (смысл которого вытекает из волнового уравнения Шрёдингера).

Применяются графические изображения распределения электронной плотности относительно ядра.

Кривая радиального распределения вероятности показывает, что электрон находится в тонком концентрическом шаровом слое радиуса r толщины dr вокруг ядра атома водорода[1].

Проекция максимума кривой соответствует боровскому радиусу α0=0,53 Å.

Во многих случаях для решения уравнения Шрёдингера используют различные приближения. Вероятностную (статистическую) интерпретацию волновой функции разработал Макс Борн. В 1954 году М.Борн удостоен Нобелевской премии по физике с формулировкой «За фундаментальные исследования в области квантовой механики, особенно, за статистическую интерпретацию волновой функции.»
}

https://ru.wikipedia.org/wiki/%D0%A1%D1%82%D0%B0%D1%82%D0%B8%D1%81%D1%82%D0%B8%D1%87%D0%B5%D1%81%D0%BA%D0%B0%D1%8F_%D0%B8%D0%BD%D1%82%D0%B5%D1%80%D0%BF%D1%80%D0%B5%D1%82%D0%B0%D1%86%D0%B8%D1%8F_%D0%B2%D0%BE%D0%BB%D0%BD%D0%BE%D0%B2%D0%BE%D0%B9_%D1%84%D1%83%D0%BD%D0%BA%D1%86%D0%B8%D0%B8
{
М. Борн вспоминал:
Он (Шрёдингер) рассматривал электрон не как частицу, но как некоторое распределение плотности, которое давалось квадратом его волновой функции |ψ|².

Он считал, что следует полностью отказаться от идеи частиц и квантовых скачков, и никогда не сомневался в правильности этого убеждения. Я, напротив, имел возможность каждодневно убеждаться в плодотворности концепции частиц, наблюдая за блестящими опытами Франка по атомным и молекулярным столкновениям, и был убеждён, что частицы не могут быть упразднены. Следовало найти путь к объединению частиц и волн. Я видел связующее звено в идее вероятности…
}




