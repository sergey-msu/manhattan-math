\chapter{Дополнение: уравнение Шреддингера}\label{ch:schroedinger_eq}


------------------------ IDEAS ------------------------ 


https://en.wikipedia.org/wiki/Hydrogen_atom
https://en.wikipedia.org/wiki/Hydrogen-like_atom
https://en.wikipedia.org/wiki/Particle_in_a_spherically_symmetric_potential
http://nuclphys.sinp.msu.ru/enc/e102.htm



...мы обещали рассказать, как находить плотность вероятности...

.
Это довольно непростой математический объект, и точные его решения известны только для простейшего случая одного единственного электрона.
 
Мы помним, что плотность вероятности обнаружения электрона в пространстве называют его \textit{орбиталью}.
Таким образом, математически, орбиталь является одним из решений уравнения Шреддингера.



УШ можно выписать не только для отдельного атома, но и для целых молекул
УШ для одного электрона решается явно.
Для двух и более электронов не решается явно.

УШ выписывается для электронов, т.е. одинаково для всех одноэлектронных (H-like) атомов и ионов.

УШ - на волновую функцию, без яркого физ смысла.

Орбитали - решения стационарного УШ

УШ - для нерелятивистского случая

https://en.wikipedia.org/wiki/Stationary_state


УШ - чисто математический объект, причем весьма непростой.

Итак, откуда же можно получить хотя бы вероятность нахождения электрона в заданной области пространства?
Оказывается, она может быть найдена точно как решение уравнения Шреддингера

\begin{equation}\label{eq:schroedinger_1}
\nabla\psi + \frac{2m}{\hbar^2}(E - U(x))\psi = 0.
\end{equation}

Решения этого уравнения $\psi(x)$ комплекснозначны, но физический смысл имеют только их модули. 
Не вдаваясь в технические детали, можно сказать, что уравнение (\ref{eq:schroedinger_1}) дает искомую плотность вероятности обнаружения электрона в точке $x$ как квадрат модуля функции-решения $|\psi(x)|^2$.
Подробнее об уравнении Шреддингера можно почитать в приложении ....
