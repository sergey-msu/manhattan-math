\chapter*{Введение}
\addcontentsline{toc}{chapter}{Введение}




Манхэттенский проект подвел выразительную черту под почти двухтысячелетними размышлениями людей о строении атома.
Он получил статус национального и по сей день является по сути единственным примером столь успешного объединения лучших умов своего времени для достижения вполне практической и понятной каждому цели \cite{bib_mathphylhist}.

В настоящее время термин ``Манхэттенский проект'' стал практически имем нарицательным, являясь синонимом концентрации научных идей и достижения конкретной прикладной цели.
Со времен середины прошлого века прошло много времени, вместившего в себя полеты человека в космос, раскрытие тайн генома человека, адронный коллайдер и многое другое, способное впечатлить даже простых людей.
Но превзойти знаменитый атомный проект США в чем-то до сих пор не удалось, хотя попытки неоднократно предпринимались.




Адронный
  , хотя попытки были.



------------------------ IDEAS ------------------------ 

Тысячелетнее стремление человека проникнуть внутрь атома
Стремление человека познать микро- и макромир. Атомизм - Греция, Индия
Стало понятно, что пришло время совершенно новой науки.
Основные достижение физики начала XX века: история
Ущемленная роль математики, причины 
Основные достижение математики начала XX века: история. Мало кому что понятно :)
Война как катализатор науки.
Второе великое объединение: физика - использует математику
В некотором роде венец и черта - ПМ
Проект "Манхэттен": [кратко]
    история
    амбициозность
    результаты
    использование лучших мировых ученых
    результаты
    влияние на науку и технику в будущем
Советский атомный проект - тоже сила мысли и пр.
Описание источников - другие книги и пр.
Описание глав
Пара слов о задачах и ответах



вырос из страха перед Германией
многие лучшие умы - беженцы из захваченных Германией территорий, некоторые потеряли семьи [описать]
ученые сами искали встечи с политиками, убеждали, что смогут построить супероружие
даже когда стало понятно, что Германия к концу войны по сути только приступила с создании супероружия ...  запущенная машина уже не могла остановиться, предостережения ученых уже не слушали
военные опасались только  одного - скорой капитуляции Японии и всеми силами стремились успеть сбросить бомбы до окончания втрой мировой войны

состав МП в итоге оказался поистине звездным [раскрыть]

Во всем обилии имеющейся информации по МП плохо прослеживается, пожалуй, лишь один мотив - роль в нем выдающихся математиков и математики как науки.
А она была колоссальна.
Сама специфика поставленной перед учеными МП задачи была такова, что новая физика атома говорила на языке математики на совершенно новом , разработанном незадолго до этого - функциональном анализе.
Более того, основной источник информации в физике - эксперименты - были просто недоступны в большом количестве из-за крайне дорогого рабочего материала - ??? - 
и крайне высокого уровня опасности проведения любых экспериментов с ним. Подобные эксперименты стоили жизни по меньшей мере двум физикам МП - ???

именно благодаря этим факторам, модели и объяснения, даваемые математиками МП, сыграли выдающуюся роль при создании сначала атомной, а затем и водородной бомб. 
Роль, которая осталась за кадром как наиболее трудно объяснимая и лишенная той прямолинейной романтики, которой обрасла роль физика-ядерщика посла второй мировой войны.

Последствия также были .... - разделы теорвера - ветвящиеся процессы, метод Монте-Карло - основа вероятностного вывода, основы вычислительной математики и кибернетики, [еще].

благодаря усилиям таких мировых величин науки как Дж. фон Нейман были созданны первые электронные вычислительные машины, роль которых и в самой войне и впоследствии переоценить сложно.
первые компьютеры были размером ...
не будет преувеличением сказать, что весь окружающий нас сегодня цифровой мир в значительной степени взял свое начало именно тогда
  
Есть еще один довольно интересный и несколько неожиданный фактор - прикладная часть математики МП была сравнительно простой. 

детали BigBoy, FatMan , детали современного автомобиля.


Спекулируя аналогией с известным в квантовой механике принципом неопределенности Гейзенберга, можно сказать, что при описании любого явления реального мира невозможно полностью уйти ни от физики явления, ни от стоящей за ней математики.
Чем более логически строже будет изложение, тем более высокий математический уровень будет требоваться от читателя.
Напротив, желание вовсе избавиться от математики скорее всего приведет к скомканному изложению, похожему на гору взятых с потолка фактов, которые предлагается принять на веру.











XX век был поистине богатым научными открытиями в самых разных областях науки. Ученые как никогда приблизились к пониманию механики как микро, так и макро процессов окружающего мира. В биологии был обнаружен и описан основной строительный блок всего живого - молекула ДНК. Стремительно начала развиваться генная инженерия, находя приложения в самых разных отраслях человеческой деятельности. В физике были открыты общая и специальная теории относительности, квантовая механика. Выдающиеся достижения физиков и биологов активно освещались в прессе и практически сразу становились предметом жаркого обсуждения даже людьми, далекими от мира науки и в лучшем случае довольно приблизительно понимающими, о чем идет речь. 
Подобного, к сожалению, нельзя сказать об отношении к достижениям математики XX века - кроме самих математиков и, пожалуй, некоторых физиков, о них не знал практически никто. А они были поистине впечатляющими, вполне сравнимыми по потенциальной мощи с квантовой механикой или открытием ДНК. Стоит упомянуть хотя бы появление и активное использование компьютеров, необходимых для сложных расчетов тогда и распространенных повсеместно сейчас.

Отсутствие должного освещения открытий математики отчасти связано с самой спецификой данной науки. Лишь в редких случаях по-настоящему сложную математическую теорию можно объяснить широкому кругу людей-непрофессионалов. Чувство красоты математических рассуждений, доказательств и окончательных выводов необходимо упорно воспитывать в себе некоторое время, прежде чем появится понимание того, что стоит за длинными формулами и придет осознание того, как полученные выводы можно применить на практике.
Данная книга призвана восполнить этот пробел и рассказать, какую роль сыграли математики в знаменитом манхэттенском проекте, явившим миру всю мощь ядерной энергии. Я попытаюсь осветить мат. аппарат, который использовался при расчетах, связанных с конструированием атомной и водородных бомб, уделяя особое внимание методам, созданным именно в процессе работы над проектом “Манхэттен”.



Формулу $E = mc^2$ как мантру может повторить практически любой современный человек.
Многие из нас так или иначе слышали о ней еще в детстве, не подозревая, что же она в действительности означает.
....


Попытки разобраться в сути какого-либо уже исследованном кем-то ранее явлении реального мира чем-то напоминают процесс очистки гипотетического фрукта с многослойной кожурой.
Первым и самым простым слоем являются личный опыт, мнения других людей и ``авторитетных'' источников о данном вопросе. 
На этом, собственно, можно и остановиться, сказав, что достаточно разобрались в вопросе.

Если полученные ответы нас не устраивают, не понятны, либо не полны и желание разобраться в сути явления не угасло, то придется перейти к следующему слою - предметной области явления, например, физике.
Необходимо хотя бы в общих чертах понять, что же именно происходит в интересующем нас явлении природы. 
Какие объекты в нем участвуют и по каким правилам взаимодействуют друг с другом. Какие моменты существенны, а какими можно пренебречь.
Продвинувшись в понимании физической сути процесса, мы 

Наконец, последний и традиционно самый трудный слой - математика явления.
Каждое явление природы имеет свой язык описания  ...  сложно .. вместо объектов - абстракции, вместо простых правил взаимодействия - сложные уравнения.





По словам Эйнштейна, ``воображение важнее знания, так как знание ограничено''.
В XX веке теории об атоме предстояло пережить... что даже воображение не всегда справлялось...


Атом вполне может приобретать или терять электроны, получая таким образом заряд и становясь \textit{ионом}.
Порцессы ионизации в ядерных реакциях нас не 



атомы - интуиция еще со времен древних греков, но дальше - перерыв почти на x000 лет связанный с тем, что увидеть объект своих измышлений уже не возможно.
прорыв - с появлением соответсвующих средств измерений, но тут ученых ждал очень большой сбрприз
до этого схема научных открытий в большинстве своем состояла в следующем - смотрели, измеряли, придумывали теорию основнную на уже известных аналогиях, потом совершенствовали приборы, и снова смотрели и придумывали анлогию и т.п.
в атомной физике известных аналогий не нашлось. Наблюдения зачастую в корне противоречили известным фактам о макромире. Любая попытка смотреть на макро-аналогии заканчивалась появлением множества противоречий теории с экспериментом и в конце концов полным провалом 


физика - ранее умение делать открытия зависело от того, насколько наблюдателен был ученый, насколько хорошо он умел проводить параллели между уже известными ялениями и только изучаемыми.
Движения огромных небесных тел описывалось исходя из аналогичных движений, которые можно было повторять в удобном масштабе в своей алборатории и т.п. [еще примеры]
Новая физика потребовала от ученых вообразить нечто не имевшее аналогов с ранее изученным в принципе. 
Это восхищало даже далеких от физики современников.
В математике такие штуки привыкли проворачивать довольно давно. 
Стефан Банах, один из творцов математики в ее современном виде, говорил: "хорошие математики видят аналогии, лучшие могут видеть аналогии между аналогиями". Сам он, безусловно, был одним из лучших.

 
слова "теория относительности", "квантовая механика" носились в воздухе. Их можно было слышать  понимающих и истолковыва

----------------------------



https://ru.wikipedia.org/wiki/%D0%AD%D0%BB%D0%B5%D0%BA%D1%82%D1%80%D0%BE%D0%BD%D0%BD%D0%B0%D1%8F_%D0%BF%D0%BB%D0%BE%D1%82%D0%BD%D0%BE%D1%81%D1%82%D1%8C
{
В качестве модели состояния электрона в атоме, в квантовой механике принято представление об электронном облаке, плотность соответствующих участков которого пропорциональна вероятности нахождения там электрона.

Электронное облако часто изображают в виде граничной поверхности. При этом обозначение электронной области при помощи точек опускают. Пространство вокруг ядра, в котором наиболее вероятно пребывание электрона, называют атомной орбиталью (смысл которого вытекает из волнового уравнения Шрёдингера).

Применяются графические изображения распределения электронной плотности относительно ядра.

Кривая радиального распределения вероятности показывает, что электрон находится в тонком концентрическом шаровом слое радиуса r толщины dr вокруг ядра атома водорода[1].

Проекция максимума кривой соответствует боровскому радиусу alpha_0=0,53 A_with_circle.

Во многих случаях для решения уравнения Шрёдингера используют различные приближения. Вероятностную (статистическую) интерпретацию волновой функции разработал Макс Борн. В 1954 году М.Борн удостоен Нобелевской премии по физике с формулировкой «За фундаментальные исследования в области квантовой механики, особенно, за статистическую интерпретацию волновой функции.»
}

https://ru.wikipedia.org/wiki/%D0%A1%D1%82%D0%B0%D1%82%D0%B8%D1%81%D1%82%D0%B8%D1%87%D0%B5%D1%81%D0%BA%D0%B0%D1%8F_%D0%B8%D0%BD%D1%82%D0%B5%D1%80%D0%BF%D1%80%D0%B5%D1%82%D0%B0%D1%86%D0%B8%D1%8F_%D0%B2%D0%BE%D0%BB%D0%BD%D0%BE%D0%B2%D0%BE%D0%B9_%D1%84%D1%83%D0%BD%D0%BA%D1%86%D0%B8%D0%B8
{
М. Борн вспоминал:
Он (Шрёдингер) рассматривал электрон не как частицу, но как некоторое распределение плотности, которое давалось квадратом его волновой функции |ψ|².

Он считал, что следует полностью отказаться от идеи частиц и квантовых скачков, и никогда не сомневался в правильности этого убеждения. Я, напротив, имел возможность каждодневно убеждаться в плодотворности концепции частиц, наблюдая за блестящими опытами Франка по атомным и молекулярным столкновениям, и был убеждён, что частицы не могут быть упразднены. Следовало найти путь к объединению частиц и волн. Я видел связующее звено в идее вероятности…
}






