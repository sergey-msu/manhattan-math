\chapter*{Введение}
\addcontentsline{toc}{chapter}{Введение}

Манхэттенский проект подвел выразительную черту под почти полувековыми размышлениями ученых о строении атома.
Он получил статус национального и по сей день является по сути единственным примером столь успешного объединения лучших умов своего времени для достижения вполне практической и понятной каждому цели \cite{bib_mathphylhist}.

имя нарицательное, синоним концентрации научных идей и достижения прикладной цели



------------------------ IDEAS ------------------------ 

Тысячелетнее стремление человека проникнуть внутрь атома
Стремление человека познать микро- и макромир. Атомизм - Греция, Индия
Стало понятно, что пришло время совершенно новой науки.
Основные достижение физики начала XX века: история
Ущемленная роль математики, причины 
Основные достижение математики начала XX века: история. Мало кому что понятно :)
Война как катализатор науки.
Второе великое объединение: физика - использует математику
В некотором роде венец и черта - ПМ
Проект "Манхэттен": [кратко]
    история
    амбициозность
    результаты
    использование лучших мировых ученых
    результаты
    влияние на науку и технику в будущем
Советский атомный проект - тоже сила мысли и пр.
Описание источников - другие книги и пр.
Описание глав
Пара слов о задачах и ответах



вырос из страха перед Германией
многие лучшие умы - беженцы из захваченных Германией территорий, некоторые потеряли семьи [описать]
ученые сами искали встечи с политиками, убеждали, что смогут построить супероружие
даже когда стало понятно, что Германия к концу войны по сути только приступила с создании супероружила ...  запущенная машина уже не могла остановиться, предостережения ученых уже не слушали
военные опасались только  одного - скорой капитуляции Японии и всеми силами стремились успеть сбросить бомбы до окончания втрой мировой войны

состав МП в итоге оказался поистине звездным [раскрыть]

Во всем обилии имеющейся информации по МП плохо прослеживается, пожалуй, лишь один мотив - роль в нем выдающихся математиков и математики как науки.
А она была колоссальна.
Сама специфика поставленной перед учеными МП задачи была такова, что новая физика атома говорила на языке математики на совершенно новом , разработанном незадолго до этого - функциональном анализе.
Более того, основной источник информации в физике - эксперименты - были просто недоступны в большом количестве из-за крайне дорогого рабочего материала - ??? - 
и крайне высокого уровня опасности проведения любых экспериментов с ним. Подобные эксперименты стоили жизни по меньшей мере двум физикам МП - ???

именно благодаря этим факторам, модели и объяснения, даваемые математиками МП, сыграли выдающуюся роль при создании сначала атомной, а затем и водородной бомб. 
Роль, которая осталась за кадром как наиболее трудно объяснимая и лишенная той прямолинейной романтики, которой обрасла роль физика-ядерщика посла второй мировой войны.

Последствия также были .... - разделы теорвера - ветвящиеся процессы, метод Монте-Карло - основа вероятностного вывода, основы вычислительной математики и кибернетики, [еще].

благодаря усилиям таких мировых величин науки как Дж. фон Нейман были созданны первые электронные вычислительные машины, роль которых и в самой войне и впоследствии переоценить сложно.
первые компьютеры были размером ...
не будет преувеличением сказать, что весь окружающий нас сегодня цифровой мир в значительной степени взял свое начало именно тогда
  
Есть еще один довольно интересный и несколько неожиданный фактор - прикладная часть математики МП была сравнительно простой. 

детали BigBoy, FatMan , детали современного автомобиля.


Спекулируя аналогией с известным в квантовой механике принципом неопределенности Гейзенберга, можно сказать, что при описании любого явления реального мира невозможно полностью уйти ни от физики явления, ни от стоящей за ней математики.
Чем более логически строже будет изложение, тем более высокий математический уровень будет требоваться от читателя.
Напротив, желание вовсе избавиться от математики скорее всего приведет к скомканному изложению, похожему на гору взятых с потолка фактов, которые предлагается принять на веру.

