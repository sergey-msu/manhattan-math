\chapter*{Введение}
\addcontentsline{toc}{chapter}{Введение}


Первая половина XX века по праву считается золотым веком физики.
В первые его десятилетия появилось все, без чего немыслима вся современная наука: общая и специальная теория относительности Эйнштейна, квантовая теория Планка, квантовая механика Бора.
Даже простое перечисление главных открытий в этих областях не может не вызывать восхищения.

Век открыла квантовая теория Макса Планка, который в 1900 году точно описал излучение абсолютно черного тела, что положило начало квантовой теории.
Основываясь на этих идеях, Альберт Эйнштейн в 1905 году объяснил фотоэффект, а Нильс Бор в 1913 году построил свою квантовую модель атома, заложив основы квантовой механики - науки, которой прочат главные открытия и в XXI веке.
Эти достижения вместе со специальной и общей теорией относительности Эйнштейна представили миру совершенно новый взгляд на окружающее нас пространство. 

Мелкие, как поначалу казалось, пробелы в давно существовавших теориях вроде отсутствия объяснения явления фотоэффекта или излучения абсолютно черного тела, открыли дверь в новую физику.
Одно за другим последовали открытия, поражающие воображение. 
Приходилось мириться с совершенно удивительными, а порой и вовсе противоречащими интуиции фактами вроде существования недостижимого предела скорости движения, зависимости результатов эксперимента от наблюдателя или невозможности точно указать местоположение частиц материи в пространстве.
Все это долго казалось фантастикой не только простым людям, но и большинству физиков того времени.

Выяснилось, что в микро- и макромасштабах время и пространство ведет себя совсем не так, как мы привыкли в обычной жизни.
Сначала теория относительности, а затем и квантовая механика сформировали новое мировоззрение, заставив одновременно с этим полностью пересмотреть основной метод получения новых знаний в физике.
Если ранее новые гипотезы формулировались на основе предыдущих наблюдений и аналогий с привычными объектами, то теперь практически единственным методом освоения реальности стало... чистое воображение.
Тот, кто мыслил нестандартно был готов отказаться от привычных взглядов на вещи, вполне мог стать одним из выдающихся ученых того времени.
Как говорил Эйнштейн, ``воображение важнее знания, так как знание ограничено''.

В начале XX века сам воздух многих европейских научных центров пропитан особым духом открытий.
Особо быделялся Геттинген - прославленная столица выдающихся математиков и физиков на протяжениии почти ... веков.
Именно там [и в Дании] образовался костяк .... и вышли.. и втом чиле Роберт Оппенгеймер - будущий научный руководитель Менхэттенскоро проекта.
 

...
Можно сказать, что , что наука в то время не нуждалась в популяризации.
Быть физиком стало модно и уважаемо..., о чем .. написано в замечательной книге Ярче тысячи солнц: ...

...
Формулу $E = mc^2$ как мантру может повторить практически любой современный человек.
Многие из нас так или иначе слышали о ней еще в детстве, не подозревая, что же она в действительности означает.
....

[атом начинал открывать свои тайны]
[с тех пор как было открыта возвожность искусстевенного расщепления атома и было замечено, что при этом выделяется огромное колическтов энергии
[мысли об использованиии этой энергии в военных целях не могла не прийти]
[по начау ее отвергали]
 отдельные ученые стали за]

[ретроспектива физики атома начала XX века - конкретно об атоме и подводка к МП]

Манхэттенский проект подвел выразительную черту под почти трехтысячелетними размышлениями людей о строении объектов окружающего мира.
Он получил статус национального и по сей день является по сути единственным примером столь успешного объединения лучших умов своего времени для достижения вполне практической и понятной каждому цели.

Со времен Второй Мировой войны прошло много времени.
История науки последних пятидесяти лет вместила в себя полет человека в космос, раскрытие тайн его генома, большой адронный коллайдер и многое другое, способное впечатлить даже не знакомых с деталями людей.
Но превзойти знаменитый атомный проект США в чем-то до сих пор не удалось, хотя попытки неоднократно предпринимались.
В настоящее время термин ``Манхэттенский проект'' стал практически именем нарицательным, являясь синонимом концентрации научных идей и блестящего достижения конкретной прикладной цели.

На реализацию проекта были выделены беспрецедентные по тем временам ресурсы - 
2 миллиарда долларов США, что эквивалентно порядка 30 миллиардам долларов в 2019 году. 
Примерно десятая часть этой суммы была потрачена на исследования, остальное - на строительство объектов и добычу радиоактивного рабочего материала.
Результат окупил себя сполна - проект был реализован в кратчайшие для таких масштабов сроки, породив самое мощное оружие в истории, еще долго являвшееся инструментом устрашения и манипуляций в руках политиков и военных разных стран.

Одним из главных достижений проекта был даже не столько конечный результат, сколько объединение ведущих ученых своего времени -  физиков и математиков, занимающихся проблемами молодой науки ядерной физики. 
Состав вовлеченных проект ученых был поистине звездным.
Практически все ученые - создатели и первопроходцы новой науки квантовой механики в той или иной мере принимали участие в создании первой атомной бомбы.
Многие из них бежали от нацистов в начале войны и имели личную мотивацию для работы над таким проектом.


..

[исключительная роль матмоделей - из Атом энерг для воен целей]

[Работа математиков]

[Вся современная математика во всей красе и мощи]
[Классика - теория меры, вероятностей]
[Новое - ветвящиеся процессы, вычислительная математика, натолкнуло на мысли об искусственном интеллекте]

[Описание глав]



------------------------ IDEAS ------------------------ 

Тысячелетнее стремление человека проникнуть внутрь атома
Стремление человека познать микро- и макромир. Атомизм - Греция, Индия
Стало понятно, что пришло время совершенно новой науки.
Основные достижение физики начала XX века: история
Ущемленная роль математики, причины 
Основные достижение математики начала XX века: история. Мало кому что понятно :)
Война как катализатор науки.
Второе великое объединение: физика - использует математику
В некотором роде венец и черта - ПМ
Проект "Манхэттен": [кратко]
    история
    амбициозность
    результаты
    использование лучших мировых ученых
    результаты
    влияние на науку и технику в будущем
Советский атомный проект - тоже сила мысли и пр.
Описание источников - другие книги и пр.
Описание глав
Пара слов о задачах и ответах



вырос из страха перед Германией
многие лучшие умы - беженцы из захваченных Германией территорий, некоторые потеряли семьи [описать]
ученые сами искали встречи с политиками, убеждали, что смогут построить супероружие
даже когда стало понятно, что Германия к концу войны по сути только приступила с создании супероружия ...  запущенная машина уже не могла остановиться, предостережения ученых уже не слушали
военные опасались только  одного - скорой капитуляции Японии и всеми силами стремились успеть сбросить бомбы до окончания второй мировой войны

состав МП в итоге оказался поистине звездным [раскрыть]

Во всем обилии имеющейся информации по МП плохо прослеживается, пожалуй, лишь один мотив - роль в нем выдающихся математиков и математики как науки.
А она была колоссальна.
Сама специфика поставленной перед учеными МП задачи была такова, что новая физика атома говорила на языке математики на совершенно новом , разработанном незадолго до этого - функциональном анализе.
Более того, основной источник информации в физике - эксперименты - были просто недоступны в большом количестве из-за крайне дорогого рабочего материала - ??? - 
и крайне высокого уровня опасности проведения любых экспериментов с ним. Подобные эксперименты стоили жизни по меньшей мере двум физикам МП - ???

именно благодаря этим факторам, модели и объяснения, даваемые математиками МП, сыграли выдающуюся роль при создании сначала атомной, а затем и водородной бомб. 
Роль, которая осталась за кадром как наиболее трудно объяснимая и лишенная той прямолинейной романтики, которой обросла роль физика-ядерщика после второй мировой войны.

Последствия также были .... - разделы теорвера - ветвящиеся процессы, метод Монте-Карло - основа вероятностного вывода, основы вычислительной математики и кибернетики, [еще].

благодаря усилиям таких мировых величин науки как Дж. фон Нейман были созданы первые электронные вычислительные машины, роль которых как в самой войне, так и впоследствии было сложно переоценить.
первые компьютеры были размером ...
не будет преувеличением сказать, что весь окружающий нас сегодня цифровой мир в значительной степени взял свое начало именно тогда
  
Есть еще один довольно интересный и несколько неожиданный фактор - прикладная часть математики МП была сравнительно простой. 

детали BigBoy, FatMan , детали современного автомобиля.


XX век был поистине богатым научными открытиями в самых разных областях науки. Ученые как никогда приблизились к пониманию механики как микро, так и макро процессов окружающего мира. В биологии был обнаружен и описан основной строительный блок всего живого - молекула ДНК. Стремительно начала развиваться генная инженерия, находя приложения в самых разных отраслях человеческой деятельности. В физике были открыты общая и специальная теории относительности, квантовая механика. Выдающиеся достижения физиков и биологов активно освещались в прессе и практически сразу становились предметом жаркого обсуждения даже людьми, далекими от мира науки и в лучшем случае лишь приблизительно понимающими, о чем идет речь. 
Подобного, к сожалению, нельзя сказать об отношении к достижениям математики XX века - кроме самих математиков и, пожалуй, некоторых физиков, о них не знал практически никто. А они были поистине впечатляющими, вполне сравнимыми по потенциальной мощи с квантовой механикой или открытием ДНК. Стоит упомянуть хотя бы появление и активное использование компьютеров, необходимых для сложных расчетов тогда и распространенных повсеместно сейчас.

Отсутствие должного освещения открытий математики отчасти связано с самой спецификой данной науки. Лишь в редких случаях по-настоящему сложную математическую теорию можно объяснить широкому кругу людей-непрофессионалов. Чувство красоты математических рассуждений, доказательств и окончательных выводов необходимо упорно воспитывать в себе некоторое время, прежде чем появится понимание того, что стоит за длинными формулами и придет осознание того, как полученные выводы можно применить на практике.
Данная книга призвана восполнить этот пробел и рассказать, какую роль сыграли математики в знаменитом манхэттенском проекте, явившим миру всю мощь ядерной энергии. Я попытаюсь осветить мат. аппарат, который использовался при расчетах, связанных с конструированием атомной и водородных бомб, уделяя особое внимание методам, созданным именно в процессе работы над проектом “Манхэттен”.






Атом вполне может приобретать или терять электроны, получая таким образом заряд и становясь \textit{ионом}.
Порцессы ионизации в ядерных реакциях нас не 



атомы - интуиция еще со времен древних греков, но дальше - перерыв почти на x000 лет связанный с тем, что увидеть объект своих измышлений уже не возможно.
прорыв - с появлением соответсвующих средств измерений, но тут ученых ждал очень большой сбрприз
до этого схема научных открытий в большинстве своем состояла в следующем - смотрели, измеряли, придумывали теорию основнную на уже известных аналогиях, потом совершенствовали приборы, и снова смотрели и придумывали анлогию и т.п.
в атомной физике известных аналогий не нашлось. Наблюдения зачастую в корне противоречили известным фактам о макромире. Любая попытка смотреть на макро-аналогии заканчивалась появлением множества противоречий теории с экспериментом и в конце концов полным провалом 




 
слова "теория относительности", "квантовая механика" носились в воздухе. Их можно было слышать  понимающих и истолковыва



