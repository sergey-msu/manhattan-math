\chapter{Самообучаемые алгоритмы}\label{ch:self_learning_algorithms}

\epigraph{\emph{.}}{}

Манхэттенский проект оказал значительное влияние на физику, математику и науку в целом.
Его населедие огромно.
Ряд подходов и целых областей современной науки родились в стенах Лос-Аламосской национальной лаборатории и затем бурно развивались, давая все новые плоды уже вне области создания ядерного оружия.

Одним из самых главных прикладных достижений проекта, сыгравшем решающую роль в его успехе, было упор на математическое и компьютерное моделирование всех необходимых физических процессов.
Это было немыслимо без мощных вычислительных машин, что в итоге определило последовавшую глобальную компьютеризацию всех областей человеческого знания.
Сначала все расчеты для Проекта велись силами целой команды \textit{human computers} - девушек-вычислителей, вручную делавших все арифметические опреации.
Руководил командой Ричард Фейнман, ставивший перед вычислителями задачи по решению сложных дифференциальных уравнений, описывающих процесс взрыва бомбы.
Очень скоро стало понятно, что уравнения сложны настолько, что использование человеческого труда становится попросту неоправданно.   
Это была поворотная точка, когда для сложных и ответственных расчетов стали использоваться компьютеры, тогда еще совсем примитивные.

К слову, некоторые из вычислителей, выполнявших до этого всю работу вручную, затем переквалифицировались для работы с новыми вычислительными машинами.
Первые шестеро программистов ЭНИАКа (первого серьезного компьютера) были именно теми самыми девушками-вычислителями [???], выполнявшими ранее всю работы 
Шесть

Развитие компьютеров стремительно развивалось сразу по нескольким направлениям.
Во-первых, развивалась сама архитектура устройств, позволяя сократить их размеры с компьютеров размером в небольшой стадион до размеров, вполне сравнимых с сегодняшними.

Во-вторых, возрастали вычислительные мощности компьютеров.
Если сначала они были сравнимы с мощностями нескольких людей-вычислителей [???], то уже через несколько лет [???]

Наконец, упрощалось само взаимодействие человека с вычислительной машиной.
Изначально выполнение даже простейших задач на компьютере было доступно лишь специалистам высочайшего класса.
Это было сделано отчасти осознанно, так как предполагалось, что подобные разработки могут пригодиться только военным, а следовательно уровень доступа к ним должен быть соответствующий.
Как говорили в свое время создательнице первого в истории компилятора*  Грейс Хоппер, “Но Грейс, ведь тогда любой сможет писать программы!”.

Сегодня компьютерные программы действительно может писать любой человек. 
Более того, подобный навык все чаще фигурирует как прямое требование при приеме на работу на самые разные должности.
Сегодня умение писать программы становится неотъемлемой частью человеческой культуры.
А истоки этого идут именно от МП.

Можно было бы много и долго перечислять конкретные достижения компьютеризации, начало которым было положено в МП.
Остановимся здесь на самой “горячей” в последние годы теме - машинном обучении и искусственном интеллекте.
Эти области на первый взгляд не имеют прямого отношения к МП.
Мы выбрали здесь именно эту область скорее для демонстрации того, какую форму могут принять правильно и вовремя поданные идеи.
Такой идеей было повсеместное использование вычислительных машин для моделирования различных процессов реального мира.

## Самообучающиеся алгоритмы

[пример: дерево и кастомные фильтры изображений]

## Проблемы обучения

[недообучение]
[переобучение - пример со студентами, реальный пример]
[обобщающая способность]

## Примеры приложений

## Общая постановка задачи

[постановка]









Новый тип алгоритмов

------------------------ IDEAS ------------------------ 

