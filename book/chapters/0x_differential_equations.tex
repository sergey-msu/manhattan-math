\chapter{Дифференциальные уравнения}\label{ch:differenrial_equations}

\epigraph{\emph{.}}{}

Уравнения являются одним из центральных объектов исследования в математике.
Составлять и решать простейшие уравнения нас учат еще со средней школе.
Там же мы знакомимся с многочисленными примерами того, где это может пригодиться.
Навыки записи и решения уравнений используется далее, в основном, в курсе школьной физики, где уравнения служат основным инструментом записи законов природы.

Не является исключением и более сложная ядерная физика, где все основные законы деления ядер записываются в виде уравнений, впрочем, куда более сложных, чем те, которые решаются в школе.
Эти уравнения носят название \textit{дифференциальных} и выражают зависимость скорости изменения некоторой величины (например, количества быстрых нейтронов в бомбе в данный момент времени) он нее самой. 
Решениями таких уравнений будут в свою очередь уже не отдельные числа, а функции, то есть зависимости одних величин от других.

Дифференциальные уравнения служат неиссякаемым источником истереснейших задач в самой математике и крайне полезны в прикладных науках. 
Они являются прочным мостом, связывающим чистую математику с прикладными науками и, в частности, теоретической физикой. 
Не будет преувеличение сказать, что вся классическая и современная физика говорят на языке дифференциальных уравнений.

В данной главе мы постараемся от простейших школьных уравнений подойти к дифференциальным, продемонстрировав их силу и суть. 


\section*{Числовые уравнения} 

Обычные уравнения, знакомые нам со школьной скамьи, обычно возникают напрямую из текстовых задач.
Обозначая неизвестные величины буквами и следуя условию задачи, мы переписываю ее в виде одного или нескольких уравнений, решая которые мы и находим все неизвестные.
Рассмотрим несколько примеров.

Пусть в процессе столкновения и последующего распада тяжелого ядра один нейтрон порождает $3$ новых нейтрона, которые далее без потерь делают то же самое.
Кроме того известно, что в системе включен постоянный источник нейтронов, дающий четыре нейтрона в единицу времени.
Требуется узнать, сколько нейтронов было в предыдущий момент, если в данный момент времени в системе имеется $100$ быстрых нейтронов.
Вводя неизвестную $x$ - число нейтронов в предыдущий момент времени, и идя по условию задачи, получаем соотношение
$$
3x + 4 = 100,
$$
представляющее собой простейшее линейное уравнение относительно неивестной $x$ с решением $x = 32$.

В процессе решения задачи вполне могут получиться и более сложные уравнения, нежели линейные.
Заметим, что в условии предыдущей задачи мы предположили, что ядерного топлива для деления у нас имеется неограниченное количество.
Такое предположение справедливо только на самых ранних этапах ядерной реакции и не имеет места, когда нейтроном становится слишком много, и топливо выгорает.
Иными словами, с течением времени не три, а все меньшее количество нейтронов будет встречаться с тяжелыми ядрами, вызывая последующий их распад.
Предположим, что топлива у нас осталось немного, и коэффициент размножения нейтронов не постоянен, а уменьшается в зависимости от числа нейтронов $x$ как $3 - 0.1x$. 
Найдем, сколько нейтронов было в системе в предыдущий момент, если в данный момент времени в системе имеется $204$ быстрых нейтрона.
Составляя уравнение согласно условию задачи, получим
$$
(3 - 0.01x)x + 4 = 104.
$$
Вспоминая школьные формулы решения квадратных уравнений, получаем $x = 100$ или $x = 200$.
Оба решения имеют физический смысл.
Без дополнительных условий мы не можм сказать, сколько нейтроном было из начально - $100$ или $200$.

Сделаем несколько важных замечаний. 
Во втором примере мы усложнили модель, включив эффект иссякания ресурсов и сделав тем самым ее более приближенной к реальности.
И сразу же усложнилось уравнение, описывающее задачу.
Оно стало нелинейным и дало два возможных решения вместо одного.
Позже мы увидим, что максимально приближенные к реальности модели вообще не могут быть решены в явном виде.
Их приходится исследовать качественно, либо численно.

Пойдем далее.
Предположим, нас интересует поведение количества нейтронов не в один фиксированный момент времени, а во все сразу.
Таким образом, интересует поведение этой величины с течением времени: будет ли число нейтронов в системе возрастать, либо убывать.
Проще говоря,... основной вопрос, который волновал ученых Манхэттенского проекта - произойдет ли взрыв?

Предположим, что мы отуда-то знаем, что коэффициент размножения участвующих в реакции нейтронов постояннен (на время пренебрежем эффектом выгорания топлива) и равен некоторому числу $k$.
Если в начальный момент времени $t=0$ в системе было $x_0$ нейтронов, то в последующий будет $kx_0$, затем $k^2x_0$ и так далее.
В момент времени $t$ количество нейтронов в системе будет равно
\begin{equation}\label{eq:simple_neutron}
x_t = k^tx_0.
\end{equation} 
Что нам дает соотношение (\ref{eq:simple_neutron})?
Во-первых, оно позволяет при известном начальном количестве нейтронов и коэффициенте размножения вычислить число нейтронов в любой момент времени.
Во-вторых, оно позволяет решать и обратные задачи: находить, каким было изначальное число нейтронов, либо коэффициент размножения, если известно их количество в другие моменты.
И наконец, в-третьих, соотношение (\ref{eq:simple_neutron}) позволяет \textit{качественно} оценить поведение в системе в целом: при $k<1$ количество нейтронов в системе будет становиться все меньше и меньше, а при $k>1$ будет неограниченно возрастать - произойдет взрыв.

Минус модели (\ref{eq:simple_neutron}) - ее неточность.
Хоть она и дает число нейтронов в любой момент времени, но это число будет не точным из-за слишком больших упрощений, сделанных при ее выводе.
Постараемся учесть эффект выгорания топлива, как это было сделано в примере 2 выше.




В школьной программе, впрочем, довольно близко подходят к этому.
Уравнения с параметром

Дифференциальные уравнения - зависимость производной величины от нее самой

История
 
